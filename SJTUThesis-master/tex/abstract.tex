%# -*- coding: utf-8-unix -*-
%%==================================================
%% abstract.tex for SJTU Master Thesis
%%==================================================

\begin{abstract}

随着云计算与大数据的快速发展,对作为其基石的存储系统提出了越来越高的要求,存储系统不仅需要具备大容量、高性能,还要满足低成本要求。而单独使用目前主流的任一种存储设备如固态盘、磁盘都不能设计出同时满足上述要求的存储系统。混合存储系统充分利用不同存储设备的互补性设计高效的存储系统,使得在保证成本不大幅上升的前提下能支持提供大容量同时具有高性能,因此成为当前的研究热点和发展方向。 

本文对混合存储系统所涉及的关键技术进行了深入研究,并实现了基于SSD和HDD的SSD缓存系统qscache,本文主要的工作如下:

\begin{enumerate}[wide]
    \item 对当前主流的存储设备的性能、成本、容量进行了比较。
    \item 对混合存储系统的关键技术,包括系统架构、数据映射策略、冷热数据识别算法、数据写回/迁移策略、最优化存储设备组合进行了深入研究。
    \item 对业界三个主流的开源混合存储系统进行了研究,比较分析了它们的性能与特性。
    \item 设计并实现了在Linux内核态运行的qscache系统,支持SSD设备和HDD设备之间的灵活映射。
    \item 对Linux内核的Device Mapper框架源代码进行修改,增加新接口,在新接口的基础上实现了qscache系统面向进程的I/O带宽按权限分配。
    \item 对qscache系统的功能与性能进行了测试,并对实验结果进行讨论分析。
\end{enumerate}

测试结果显示qscache系统的整体性能与flashcache接近,并且能够实现多块SSD设备与多块HDD设备之间的映射,对Linux内核的修改新增的接口能够支持qscache系统以基于request的模式运行,在此基础上支持对I/O带宽按权限分配的功能,本论文的研究成果对于其它的混合存储系统设计与实现具有较好的借鉴价值。

\keywords{\large 固态盘 \quad 磁盘 \quad 分层 \quad 缓存}
\end{abstract}

\begin{englishabstract}

With the rapid development of cloud computing and big data, the storage system, which is the cornerstone, needs to provide high storage capacity and high performance at low cost. While at present any one of the storage devices, such as solid state disk and hard disk drive, can not meet the above requirements alone. Hybrid storage system takes full advantage of the characteristics of different storage device to support large-capacity, but also ensures high performance without too much extra cost. Therefore, hybrid storage system is becoming current research focus and research direction. 

This paper conducted in-depth research on the key technologies in hybrid storage system. Then the SSD cache system based on SSD and HDD, qscache was designed and implemented. The main work of this paper is as follows:


\begin{enumerate}[wide]
    \item The performance, cost and capacity of current wide-used storage devices were analyzed and compared.
    \item Several key technologies including system architecture, data mapping strategy, hot data identification strategy, data write back/migration strategy, optimal combination of storage devices in  hybrid storage systems were studied deeply.
    \item Three widely-used open source hybrid storage systems were studied, and their performance and characteristics were analyzed and compared.
    \item The design and implementation of hybrid storage system qscache including the support of flexible mapping from SSD to HDD.
    \item The source code of Device Mapper framework in Linux kernel was modified to support a new interface, based on which, the qscache system supports assigning IOPS for different processes based on weight.
    \item The performance of the qscache system was tested, and the results were analyzed.
\end{enumerate}

The test results shows that the performance of qscache system was close to flashcache, and mapping between several SSDs and HDDs was implemented. The modification of Linux kernel supports the request-based mode qscache system, based on which assigning IOPS for different processes based on weight was supported. This research provides a reference for the design and implementation of other hybrid storage systems.

\englishkeywords{\large  solid state disk, hard disk drive, multi-layer, cache}
\end{englishabstract}

