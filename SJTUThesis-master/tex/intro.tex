%# -*- coding: utf-8-unix -*-
%%==================================================
%% chapter01.tex for SJTU Master Thesis
%%==================================================

%\bibliographystyle{sjtu2}%[此处用于每章都生产参考文献]
\chapter{绪论}
\label{chap:intro}

\section{研究背景与意义}

\subsection{研究背景}
\label{sec:backgrounds}

大规模分布式存储系统是是云计算与大数据的基石,是近年来的研究热点。随着近年来云计算与大数据的兴起,现有的存储设备正面临着诸多挑战。首先,随着数据量的爆发式增长,对存储系统的容量提出了更高的要求,如何在提供大容量存储的前提下尽可能降低成本是一大挑战\cite{gray2003next}。其次,由于云计算中的多台虚拟机可能会依赖于同一存储系统,这就导致该存储系统的负载整体会表现为随机化与碎片化,这对存储系统的随机读写性能有很高要求。最后,至今计算性能可谓日益提高,相比最初已有了巨大的提升,但存储设备的读写性能提高仍然有限,这就导致计算性能与存储性能之间的差距越来越巨大\cite{morris2003evolution},如何能提供与计算性能相匹配的存储设备的读写性能也是另一大挑战。

根本上来看,存储系统的性能极大取决于存储系统所基于的存储设备。目前,磁盘(Hard Disk Drive:HDD)仍然是存储系统中最主要的存储设备,虽然随着各种硬件技术的发展,磁盘的容量依然依照摩尔定律\cite{schirle1996history}在过去的30多年中增长了近十万倍,但是由于其磁头是机械移动这一根本性的限制,磁盘的访问延迟在过去的30年中仅提高了约2倍,而如果通过提高转速来进一步提高磁盘的访问性能,过高的转速会引发能耗与散热等诸多问题,因此磁盘的容量与访问性能之间的鸿沟十分巨大。

近年来,SSD(Solid State Disk, 也被称为固态盘)的快速发展为提升存储系统的性能带来了新的研究方向。与磁盘不同,SSD是电子器件,不像磁盘需要磁头的机械移动,因此SSD的随机读写性能要远胜于磁盘,另外由于没有磁头的机械移动,SSD相比磁盘的能耗更低,另外也更具抗震性。然而SSD仍有诸多问题。首先,虽然SSD相比磁盘在随机读写上性能要优秀得多,然而在顺序读写上性能相比磁盘并没有明显的优势。其次,SSD的成本相比磁盘仍要高出许多。最后,SSD的读写性能并不对称,这是由于SSD的写操作造成的,SSD对数据进行更新时需要首先将新数据写入空闲页面,然后将原页面标记为invalid,之后等待将原页面擦除后再将新数据写入原页面,最终导致SSD的写操作耗时约为读操作的6倍,并且由于SSD的擦除操作具有约五万次的次数限制,超出次数后的页面将无法使用,因此会带来耐久性问题。虽然SSD有上述几个问题,但SSD的优缺点与磁盘的优缺点可以很好地互补,因此将SSD与磁盘结合使用为设计大容量、高性能、低成本的存储系统提供了新的可能。

\subsection{研究意义}

目前在学术界与产业界,混合存储系统的研究越发受到重视,各方面的研究都有在开展。但是对于混合存储系统的研究仍然有些问题,主要体现在两个方面:

一方面,总的来看,学术界目前对于混合存储系统的研究尚处于起步阶段,很多工作的重点并不在于混合存储系统的设计与实现,而主要在关注SSD的物理结构与特性,关于基于SSD与磁盘的混合存储系统的文献\cite{guerra2011cost, kim2011hybridstore}能公开搜索到的很少。另外在少数提出的基于SSD与磁盘的混合存储系统的文献中,真正进行了测试验证的文献更少,更多的是仅仅提供一种思路,无法验证其正确性。产业界中一些国际一流的存储设备厂商推出的存储产品中已经有开始使用基于SSD与磁盘的混合存储设计,但由于缺乏公开的文献资料,这些存储设备的内部实现无从知晓,性能的提升仅能依赖于一些简单的测试而并不能得知各项具体的性能提升。

另一方面,由于之前提到的SSD的擦除次数限制,这就导致在混合存储系统的写操作面临两难的境地:如果对于数据的写操作都在SSD中进行,那么势必会影响SSD的寿命,而如果将数据的写操作放到磁盘进行,那么势必会影响混合存储系统整体的写性能。因此目前混合存储系统的设计中仍有问题尚待解决。

因此,把握住SSD这一新型存储器件带来的机遇,开展基于SSD与磁盘的混合存储系统的研究,一方面是对于存储系统基础理论的创新,另一方面也对提升我国存储领域的核心竞争力具有重大意义。更进一步,也是为当下以及未来,我国云计算与大数据的发展打下坚实的存储基础。

\section{研究内容与目标}

本文将通过学习存储相关理论,并结合现有的混合存储系统实例分析,研究混合存储系统中的关键技术,在此基础上设计实现以SSD作为磁盘缓存的SSD缓存系统,对其进行测试、完善和优化,并与其它软件进行比较测试。

本文主要将包括以下三大目标:
\begin{enumerate}
    \item SSD、磁盘的随机读写性能、顺序读写性能、容量、成本的对比分析。
    \item 现有混合存储系统的随机读写性能、顺序读写性能、可靠性、额外开销的对比分析。
    \item 混合存储系统设计中的关键技术研究。
    \item 设计并实现一个基于SSD与磁盘的混合存储系统
\end{enumerate}

\section{论文结构}

论文正文分为七个章节,各章的内容安排如下: 

第一章:绪论。给出了论文的研究背景及意义,阐述了论文课题的研究内容和目标,最后说明了论文的组织结构。 

第二章:混合存储系统介绍。介绍了混合存储系统的概念,对当前主流的存储设备的特性进行了介绍,并对混合存储系统的发展历程进行了介绍。 

第三章:混合存储系统设计中的关键技术研究。本章节从系统架构、数据映射策略、冷热数据识别策略、数据写回/迁移策略、最优化存储设备组合这五个方面展开分析介绍混合存储系统设计中的关键技术。 

第四章:开源混合存储系统介绍。本章节对于目前主流的三个开源的混合存储框架进行了对比分析。

第五章:系统设计。首先介绍了Trident系统的设计目标与设计思想,然后介绍了Trident系统的系统架构、数据映射策略、冷热数据识别策略、数据写回/迁移策略、最优化存储设备组合等设计。 

第六章:系统实现。首先对于系统实现过程中需要掌握的预备知识进行介绍,然后对系统中一些关键问题的具体实现方法进行阐述。 

第七章:系统测试。首先介绍了系统测试的目标与测试方法,然后介绍了测试环境与系统初始化过程,之后进行测试,最后进行了小结。 

第八章:总结与展望。对论文所做的工作进行了总结,并且展望了将来可能的改进工作与改进方向。

