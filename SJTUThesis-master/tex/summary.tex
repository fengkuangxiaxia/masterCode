%# -*- coding: utf-8-unix -*-
%%==================================================
%% conclusion.tex for SJTUThesis
%% Encoding: UTF-8
%%==================================================

%\begin{summary}

\chapter{总结与展望}
\label{chap:summary}

\section{全文总结}

本文针对基于SSD与HDD的混合存储系统,首先介绍了混合存储系统的基本概念和发展历史,然后研究了混合存储系统在设计中的关键技术,研究了三个开源的混合存储系统flashcache、dm-cache和bcache,分析对比了它们的特性和性能,然后设计并实现了基于SSD和HDD的混合存储系统qscache,最后对系统的性能进行了测评,并对系统对多缓存设备对多后台设备以及对I/O带宽按权限分配的功能进行了测试验证。具体工作如下:

\begin{enumerate}
    \item 介绍了混合存储系统产生的原因以及基本概念,并对当前主流的存储设备进行了对比,介绍了混合存储系统的发展历史。
    \item 介绍了混合存储系统在设计中的关键技术,包括系统架构、数据映射策略、冷热数据识别策略、数据写回/迁移策略、最优化存储设备组合这五大方面。
    \item 介绍了三个开源混合存储系统flashcache、dm-cache和bcache并对它们的性能和特性进行了分析对比。
    \item 详细介绍了qscache系统的设计。首先介绍了系统的设计动机和设计目标然后对qscache系统在混合存储系统中的关键技术展开介绍,另外对按权限分配I/O带宽的功能的设计进行了介绍。
    \item 详细介绍了qscache系统的实现。首先对系统实现过程中必要的预备知识进行了简单的介绍,然后对系统的编程实现进行了详细的介绍。
    \item 对qscache系统进行了测试。首先对比分析了基于HDD的系统、flashcache和qscache在顺序读写性能和随机读写性能上的差异,然后验证了qscache对多缓存设备对多后台设备以及对I/O带宽按权限分配的功能的实现,最后对测试结果进行了分析。
\end{enumerate}

\section{研究展望}

本文虽然已经取得了一些研究成果,但仍然存在可以进一步改进的内容:

\begin{enumerate}
    \item 目前qscache系统支持按权限分配不同的I/O带宽,但是仅仅只是限制了不同权限的进程能被分配到的IOPS不同,而对混合存储系统而言最重要的资源是缓存,目前qscache系统尚不支持针对不同权限的进程能按权限比例限制进程占有的缓存空间。在接下来的研究中考虑针对不同的权限分别维护一个LRU链表,每次有进程启动则针对该进程的权限慢慢将分配给它的缓存块划入对应的LRU链表管理,不同LRU链表按照不同权限设置不同的最大长度,这样不同权限的进程能得到的最大缓存块数量就实现了按权限分配。
    \item 目前qscache系统虽然支持多缓存设备对多后台设备,但是仍然是静态配置,需要在系统初始化时进行配置,而不能在系统运行过程中动态地增加缓存设备或后台设备。这个问题比较复杂,动态地扩容不仅影响设备的管理也会影响数据映射、冷热数据识别等多个方面,需要在之后的研究中进一步改进。
    \item 目前qscache系统虽然性能基本能达到flashcache的水平,但性能并没有达到理论极限,因此在后续研究中需要针对数据映射策略、冷热数据识别策略、数据写回/迁移策略这几个策略进行改进,进一步提升系统性能。
    \item 目前qscache仍只是单台机器的混合存储系统,而不支持分布式存储下的混合存储,后续研究中可以针对分布式存储环境下的混合存储技术展开进一步研究。
\end{enumerate}

%\end{summary}
